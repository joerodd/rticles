%%%%%%%%%%%%%%%%%%%%%%%%%%%%%%%%%%%%%%%%%%%%%%%%%%%%%%%
%	JASA-EL LaTeX Sample File
%
%  Beginner Latex users should refer to their favorite online documentation
%  here is one from the TeX Users Group 
%	https://www.tug.org/twg/mactex/tutorials/ltxprimer-1.0.pdf
%
%  Useful FAQ from  https://journals.aps.org/revtex/revtex-faq
% 
%%%%%%%%%%%%%%%%%%%%%%%%%%%%%%%%%%%%%%%%%%%%%%%%%%%%%%%

%%%%%%%%%%%%%%%%%%%%%%%%%%%%%%%%%%%%%%%%%%%%%%%%%%%%%%%
%%%%% Track Changes %%%%%
%% The track changes option allows you to mark changes
%% and will produce a list of changes, their line number
%% and page number at the end of the article.

%% Use this option for Track Changes:
\documentclass[trackchanges]{JASA-EL}

%%%%%%% JASA-EL without options %%%%%%%%%%%%%%%%%%%%%%%
%% For manuscript, 12pt, one column style, line numbers default

% \documentclass{JASA-EL}

%%%%% JASA-EL Options %%%%%
%% 
%% Remember only one documentclass command may  be used, so
%% if you use one of these document class commands with options,
%% remember to comment out the other instances of \documentclass

%%%%%%%%%%%%%%%%%%%%%%%%%%%%%%%%%%%%%%%%%%%%%%%%%%%%%%%%
%% The default option produces all authors, all affiliations,
%% then all email addresses, ie,

%%
% author one, author two, author three, author four, author five
%%
% affiliation,  affiliation, affiliation
%%
% email for author one, email for author two, email for author three,
% email for author four, email for author five
%%


%%%%%%%%%%%%%%%%%%%%%%%%%%%%%%%%%%%%%%%%%
%% Author Affiliation Option 
% \documentclass[authaffil]{JASA-EL}

% authaffil option will produce all email addresses for previously
% mentioned authors immediately after affiliation; ie,

% author one, author two, author three
% affiliation
% email for author one, email for author two, email for author three
% 
% author four
% affiliation
% email
% 
% author five
% affiliation
% email for author five

%%%%%%%%%%%%%%%%%%%%%%%%%%%%%%%%%%%%%%%%%
%% Reference Options:

%% Default is Author-Year style.
% \documentclass{JASA-EL}

%% NumberedRefs Option
%% The NumberedRefs option produces numbered bibliography and citations.
% \documentclass[NumberedRefs]{JASA-EL}

%%%%%%%%%%%%%%%%%%%%%%%%%%%%%%%%%%%%%%%%%
%% Line numbers, on by default. Here's how to turn them off:
%% \documentclass[turnofflinenumbers]{JASA-EL}

\begin{document}

%% Version in square brackets will appear at bottom right of title page:
\title[JASA-EL/Sample JASA-EL Article]{Sample JASA-EL Article}
\author{Author One}
\email{author.one@someplace.edu}
\author{Author Two}
\email{author.two@someplace.edu}
\author{Author Three}
\email{author.three@someplace.edu}
\affiliation{Department1,  University1, City, State ZipCode,
Country}
\author{Author Four}
\email{author.four@someplace.edu}
%% This command must be used for corresponding author:
\correspondingauthor
%%
\affiliation{Department2,  University2, City, State ZipCode, Country}
\author{Author Five}			
\email{author.five@someplace.edu}
\affiliation{Department3,  University3, City, State ZipCode, Country}

\date{\today} 

\begin{abstract}
Put your abstract here. Abstracts are limited to 100 words for
JASA-EL
articles. Please no
personal pronouns, also please do not use the words ``new'' and/or
``novel'' in the abstract. An article usually includes an abstract, a
concise summary of the work covered at length in the main body of the
article.     
\end{abstract}

%% pacs numbers not used

\maketitle

%  End of title page for Preprint option --------------------------------- %

\section{\label{sec:1} Introduction section}
This sample document demonstrates the use of JASA-EL in manuscripts 
prepared for submission to the Journal of the Acoustical Society of America.  
See JASA-EL-Docs.pdf, which is part of this package, for extensive
documentation on using commands for JASA-EL.

You can compare the .tex version of this file with the resulting .pdf
version to give you an idea of what  commands are available and how
they work. At the top of the .tex file you'll find a listing of the
documentclass options, and an explanation of their results.
Some additional suggestions are included in the body of this
manuscript.  

  Beginner Latex users should refer to their favorite online
  documentation. An
  useful place to start is the primer from the TeX Users Group 
  \url{https://www.tug.org/twg/mactex/tutorials/ltxprimer-1.0.pdf}


\clearpage


%%%%%%%%%%%%%%%%%%%%%%%%%%%%%%%%%%%%%%%%%%%%%%%%%%%%%%%%%%%%%%%%%%%%%
% Track Changes:
% To add words, \added{<word added>}
% To delete words, \deleted{<word deleted>}
% To replace words, \replace{<word to be replaced>}{<replacement word>}
% To explain why change was made: \explain{<explanation>}

% At the end of the document, use \listofchanges, which will list the
% changes and the page and line number where the change was made.

% When final version, \listofchanges will not produce anything,
% \added{} word will be printed, \deleted{} will take away the word,
% \replaced{}{} will print only the 2nd argument.
% \explain will not print anything.
%%%%%%%%%%%%%%%%%%%%%%%%%%%%%%%%%%%%%%%%%%%%%%%%%%%%%%%%%%%%%%%%%%%%%

%% Track changes does not work unless `trackchanges' is used as an optional argument:
%% \documentclass[trackchanges]{JASA-EL}

\section{Track Changes}
ASA prefers that the Track Changes commands only be used to
track revisions.

\subsection{Using track changes commands}
Track changes commands will work only when the option
\verb+\trackchanges+ is used:\\
\verb+\documentclass[trackchanges]{JASA-EL}+.

\subsection{Available track changes commands}

To add words, \verb+\added{<word added>}+

To delete words, \verb+\deleted{<word deleted>}+

To replace words, \verb+\replace{<word to be replaced>}{<replacement word>}+

To explain why change was made: \verb+\explain{<explanation>}+


\subsection{Available option for track changes commands}

Comments can be used for additional information to the author, perhaps
a date, or the editor's initials, or more text.
\vskip12pt
To add comment when adding words, \verb+\added[comment]{<word added>}+

To delete words, \verb+\deleted[comment]{<word deleted>}+

To replace words, \verb+\replace[comment]{<word to be replaced>}{<replacement word>}+

To explain why change was made: \verb+\explain[comment]{<explanation>}+


\subsection{End of document, `list of changes'}

At the end of the document, use \verb+\listofchanges+ is called by
\verb+\end{document}+. It will list the
changes and the page and line number where the change was made.

When final version, \verb+\listofchanges+ will not produce anything,
\verb+\added{}+ word will be printed, \verb+\deleted{}+ will take away the word,
\verb+\replaced{}{}+ will print only the 2nd argument.
\verb+\explain{}+ will not print anything.

%% only use these samples if the `trackchanges' option has been used:

\subsection{Samples of track changes}

This shows `added': \added{This was added to the text}

%%
%% Use \explain{} before change, so that it ends up on the right line.
\explain{Redundant sentence, better without it. Do you mind?\\ --~JC}

Here is an example of deleted in the body  of a paragraph.\deleted{This was deleted from the text}

We replaced \replaced{XYZ}{ZYX}

At the end of the document, use \verb+\listofchanges+ is called by
\verb+\end{document}+. It will list the
changes and the page and line number where the change was made.

If `trackchanges' is not an option, \verb+\listofchanges+ will
not produce anything. \verb+\added{xyz}+ will put `xyz' in the
text; \verb+\delete{zzz}+ will produce nothing; and 
\verb+\replace{abc}{def}+ will leave `def' in your text.

\subsection{Add comment for the change?}
If you want to add a name to identify who made the change,
or any other comment, you can use  \verb+[]+ to enter a comment,
i.e.,\\ \verb+\added[Amy, Feb 2, 2017]+\\
\verb+{This was added to the text}+.

\added[Amy, Feb 2, 2017]{This was added to the text}

\deleted[Not really necessary]{This was deleted from the text}

We replaced \replaced[(written backwards originally)]{XYZ}{ZYX}



\end{document}




